\documentclass[10pt, letterpaper]{article}
% Packages:
\usepackage[
    ignoreheadfoot, % set margins without considering header and footer
    top=2 cm, % seperation between body and page edge from the top
    bottom=2 cm, % seperation between body and page edge from the bottom
    left=2 cm, % seperation between body and page edge from the left
    right=2 cm, % seperation between body and page edge from the right
    footskip=1.0 cm, % seperation between body and footer
    % showframe % for debugging 
]{geometry} % for adjusting page geometry
\usepackage{titlesec} % for customizing section titles
\usepackage{tabularx} % for making tables with fixed width columns
\usepackage{array} % tabularx requires this
\usepackage[dvipsnames]{xcolor} % for coloring text
\definecolor{primaryColor}{RGB}{0, 0, 0} % define primary color
\usepackage{enumitem} % for customizing lists
\usepackage{fontawesome5} % for using icons
\usepackage{amsmath} % for math

%%My Commands
    \usepackage[Portugues]{optional}
        \newcommand{\optPortugues}[1]{\opt{Portugues}{#1}}
    % \usepackage[English]{optional}
        \newcommand{\optEnglish}[1]{\opt{English}{#1}}    

    \newcommand{\multilanguage}[2]{\optPortugues{#1}\optEnglish{#2}}
    \newcommand{\boldAndPlain}[2]{\textbf{#1}: #2}
    \newcommand{\highlightItem}[2]{\item\boldAndPlain{#1}{#2}}
%%

\usepackage[
    pdftitle={NOME E SOBRENOME},
    pdfauthor={NOME E SOBRENOME},
    pdfcreator={LaTeX with RenderCV},
    colorlinks=true,
    urlcolor=primaryColor
]{hyperref} % for links, metadata and bookmarks
\usepackage[pscoord]{eso-pic} % for floating text on the page
\usepackage{calc} % for calculating lengths
\usepackage{bookmark} % for bookmarks
\usepackage{lastpage} % for getting the total number of pages
\usepackage{changepage} % for one column entries (adjustwidth environment)
\usepackage{paracol} % for two and three column entries
\usepackage{ifthen} % for conditional statements
\usepackage{needspace} % for avoiding page brake right after the section title
\usepackage{iftex} % check if engine is pdflatex, xetex or luatex

% Ensure that generate pdf is machine readable/ATS parsable:
\ifPDFTeX
    \input{glyphtounicode}
    \pdfgentounicode=1
    \usepackage[T1]{fontenc}
    \usepackage[utf8]{inputenc}
    \usepackage{lmodern}
\fi

\usepackage{charter}

% Some settings:
\raggedright
\AtBeginEnvironment{adjustwidth}{\partopsep0pt} % remove space before adjustwidth environment
\pagestyle{empty} % no header or footer
\setcounter{secnumdepth}{0} % no section numbering
\setlength{\parindent}{0pt} % no indentation
\setlength{\topskip}{0pt} % no top skip
\setlength{\columnsep}{0.15cm} % set column seperation
\pagenumbering{gobble} % no page numbering

\titleformat{\section}{\needspace{4\baselineskip}\bfseries\large}{}{0pt}{}[\vspace{1pt}\titlerule]

\titlespacing{\section}{
    % left space:
    -1pt
}{
    % top space:
    0.3 cm
}{
    % bottom space:
    0.2 cm
} % section title spacing

\renewcommand\labelitemi{$\vcenter{\hbox{\small$\bullet$}}$} % custom bullet points
\newenvironment{highlights}{
    \begin{itemize}[
        topsep=0.10 cm,
        parsep=0.10 cm,
        partopsep=0pt,
        itemsep=0pt,
        leftmargin=0 cm + 10pt
    ]
}{
    \end{itemize}
} % new environment for highlights


\newenvironment{highlightsforbulletentries}{
    \begin{itemize}[
        topsep=0.10 cm,
        parsep=0.10 cm,
        partopsep=0pt,
        itemsep=0pt,
        leftmargin=10pt
    ]
}{
    \end{itemize}
} % new environment for highlights for bullet entries

\newenvironment{onecolentry}{
    \begin{adjustwidth}{
        0 cm + 0.00001 cm
    }{
        0 cm + 0.00001 cm
    }
}{
    \end{adjustwidth}
} % new environment for one column entries

\newenvironment{twocolentry}[2][]{
    \onecolentry
    \def\secondColumn{#2}
    \setcolumnwidth{\fill, 4.5 cm}
    \begin{paracol}{2}
}{
    \switchcolumn \raggedleft \secondColumn
    \end{paracol}
    \endonecolentry
} % new environment for two column entries

\newenvironment{threecolentry}[3][]{
    \onecolentry
    \def\thirdColumn{#3}
    \setcolumnwidth{, \fill, 4.5 cm}
    \begin{paracol}{3}
    {\raggedright #2} \switchcolumn
}{
    \switchcolumn \raggedleft \thirdColumn
    \end{paracol}
    \endonecolentry
} % new environment for three column entries

\newenvironment{header}{
    \setlength{\topsep}{0pt}\par\kern\topsep\centering\linespread{1.5}
}{
    \par\kern\topsep
} % new environment for the header

\newcommand{\placelastupdatedtext}{% \placetextbox{<horizontal pos>}{<vertical pos>}{<stuff>}
  \AddToShipoutPictureFG*{% Add <stuff> to current page foreground
    \put(
        \LenToUnit{\paperwidth-2 cm-0 cm+0.05cm},
        \LenToUnit{\paperheight-1.0 cm}
    ){\vtop{{\null}\makebox[0pt][c]{
        \small\color{gray}\textit{Last updated in September 2024}\hspace{\widthof{Last updated in September 2024}}
    }}}%
  }%
}%

% save the original href command in a new command:
\let\hrefWithoutArrow\href

% new command for external links:

\begin{document}
    \newcommand{\AND}{\unskip
        \cleaders\copy\ANDbox\hskip\wd\ANDbox
        \ignorespaces
    }
    \newsavebox\ANDbox
    \sbox\ANDbox{$|$}

    \begin{header}
        \fontsize{25 pt}{25 pt}\selectfont NOME E SOBRENOME
        \vspace{5 pt}

        \normalsize
        \mbox{São Paulo, \multilanguage{Brasil}{Brazil}}%
        \kern 5.0 pt%
        \AND%
        \kern 5.0 pt%
        \mbox{\hrefWithoutArrow{mailto:xxxxxxx@gmail.com}{xxxxxxx@gmail.com}}%
        \kern 5.0 pt%
        \AND%
        \kern 5.0 pt%
        \mbox{\hrefWithoutArrow{tel:+55-11-99123-4567}{(11) 99123-4567}}%
        \kern 5.0 pt%
        \AND%
        \kern 5.0 pt%
        \mbox{\hrefWithoutArrow{linkedin.com/in/exemplo/}{linkedin.com/in/exemplo/}}%
    \end{header}

    \vspace{5 pt - 0.3 cm}
    \section{
        \opt{Português}{Experiências Profissionais}
        \opt{English}{Professional Experiences}
    }       
    %%Condor
        \begin{twocolentry}{mm/yyyy – \multilanguage{Atual}{Current}}
            \boldAndPlain{\multilanguage
                {Posição Atual}
                {Current Position}}
            {Empresa 1, Cidade, \multilanguage{Brasil}{Brazil} }
        \end{twocolentry}

        \vspace{0.10 cm}

        \begin{onecolentry}
            \textit{\multilanguage
            {Descrição das atividades da empresa atual}
            {Description of the activities of current company}}
        \end{onecolentry}

        \begin{onecolentry}
            \textit{\multilanguage
                {Detalhes da posição atual}
                {Details related to current position}}
        \end{onecolentry}

        \begin{highlights}
            \item{\multilanguage
                {Ponto de destaque 1 da posição atual}
                {Highlight 1 of current position}}
            \item{\multilanguage
                {Ponto de destaque 2 da posição atual}
                {Highlight 2 of current position}}
        \end{highlights}

        \vspace{0.5 cm}
    
    %% MODCO
        \begin{twocolentry}{mm/yyyy – mm/yyyy}
            \boldAndPlain{\multilanguage
                {Posição anterior 1}
                {Previous position 1}}
            {Empresa 2, Cidade, \multilanguage{Brasil}{Brazil} }
        \end{twocolentry}
        
        \vspace{0.10 cm}
        
        \begin{onecolentry}
            \textit{\multilanguage
            {Descrição das atividades da empresa anterior}
            {Description of the activities of previous company}}
        \end{onecolentry}
        
        \vspace{0.10 cm}
        
        \begin{onecolentry}
            \textit{\multilanguage
                {Detalhes da posição anterior}
                {Details of previous position}}
        \end{onecolentry}

        \vspace{0.2 cm}

        \begin{highlights}
            \item{\multilanguage
                {Ponto de destaque 1 da posição anterior}
                {Highlight 1 of previous position}}
            \item{\multilanguage
                {Ponto de destaque 2 da posição anterior}
                {Highlight 2 of previous position}}
        \end{highlights}

    \section{\multilanguage{Tecnologias}{Technologies}}
    \begin{highlights}
        \highlightItem{\multilanguage
            {Item 1}
            {Item 1}}
        {\multilanguage
            {Detalhes do item 1}
            {Details of item 1}}
        \highlightItem{\multilanguage
            {Item 2}
            {Item 2}}
        {\multilanguage
            {Detalhes do item 2}
            {Details of item 2}}
    \end{highlights}
    \section{\multilanguage
        {Publicações}
        {Publications}}        
        \begin{samepage}
            \begin{twocolentry}{
                mm/yyyy
            }
            \boldAndPlain{\multilanguage{Onde foi publicado}{Where it was published}}{{\href{https://google.com}{\multilanguage
                {Título da publicação 1}
                {Title of publication 1}}}}
            \end{twocolentry}
            \vspace{0.10 cm}            
        \end{samepage} 

    \section{\multilanguage
        {Formação}
        {Education}}        
        \begin{twocolentry}{
            yyyy – yyyy
        }
        \boldAndPlain{Instituição 1}
            {\multilanguage
            {Curso 1}
            {Course 1}
        }
        \end{twocolentry}
        
        \vspace{0.10 cm}
        
        {\multilanguage
            {Detalhes do curso 1}
            {Details of course 1}}        

        \vspace{0.50 cm}

        \begin{twocolentry}{
            yyyy - yyyy
        }
        \boldAndPlain{Instituição 2}
            {\multilanguage
            {Curso 2}
            {Course 2}
        }
        \end{twocolentry}
        
        \vspace{0.10 cm}
        
        {\multilanguage
            {Detalhes do curso 2}
            {Details of course 2}} 
            
    \section{\multilanguage
        {Idiomas}
        {Languages}}
        \begin{highlights}
            \highlightItem{\multilanguage
                {Idioma 1}
                {Language 1}}
            {\multilanguage
                {Detalhes Idioma 1}
                {Details about language 1}}
            \highlightItem{\multilanguage
                {Idioma 2}
                {Language 2}}
            {\multilanguage
                {Detalhes Idioma 2}
                {Details about language 2}}
        \end{highlights}
        
    \section{
        \optEnglish{Personal Interests}
        \optPortugues{Interesses Pessoais}
    }
        \begin{highlights}
            \highlightItem{\multilanguage
                {Interesse 1}
                {Interest 1}}
            {\multilanguage
                {Descrição interesse 1}
                {Description for interest 1}}
            \highlightItem{\multilanguage
                {Interesse 2}
                {Interest 2}}
            {\multilanguage
                {Descrição interesse 2}
                {Description for interest 2}}
        \end{highlights}
\end{document}